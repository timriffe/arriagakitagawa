%Version 2.1 April 2023
% See section 11 of the User Manual for version history
%
%%%%%%%%%%%%%%%%%%%%%%%%%%%%%%%%%%%%%%%%%%%%%%%%%%%%%%%%%%%%%%%%%%%%%%
%%                                                                 %%
%% Please do not use \input{...} to include other tex files.       %%
%% Submit your LaTeX manuscript as one .tex document.              %%
%%                                                                 %%
%% All additional figures and files should be attached             %%
%% separately and not embedded in the \TeX\ document itself.       %%
%%                                                                 %%
%%%%%%%%%%%%%%%%%%%%%%%%%%%%%%%%%%%%%%%%%%%%%%%%%%%%%%%%%%%%%%%%%%%%%

\documentclass[sn-apa,pdflatex]{sn-jnl}

%%%% Standard Packages
%%<additional latex packages if required can be included here>

\usepackage{graphicx}%
\usepackage{multirow}%
\usepackage{amsmath,amssymb,amsfonts}%
\usepackage{amsthm}%
\usepackage{mathrsfs}%
\usepackage[title]{appendix}%
\usepackage{xcolor}%
\usepackage{textcomp}%
\usepackage{manyfoot}%
\usepackage{booktabs}%
\usepackage{algorithm}%
\usepackage{algorithmicx}%
\usepackage{algpseudocode}%
\usepackage{listings}%
%%%%

%%%%%=============================================================================%%%%
%%%%  Remarks: This template is provided to aid authors with the preparation
%%%%  of original research articles intended for submission to journals published
%%%%  by Springer Nature. The guidance has been prepared in partnership with
%%%%  production teams to conform to Springer Nature technical requirements.
%%%%  Editorial and presentation requirements differ among journal portfolios and
%%%%  research disciplines. You may find sections in this template are irrelevant
%%%%  to your work and are empowered to omit any such section if allowed by the
%%%%  journal you intend to submit to. The submission guidelines and policies
%%%%  of the journal take precedence. A detailed User Manual is available in the
%%%%  template package for technical guidance.
%%%%%=============================================================================%%%%

%% Per the spinger doc, new theorem styles can be included using built in style, 
%% but it seems the don't work so commented below
%\theoremstyle{thmstyleone}%
\newtheorem{theorem}{Theorem}%  meant for continuous numbers
%%\newtheorem{theorem}{Theorem}[section]% meant for sectionwise numbers
%% optional argument [theorem] produces theorem numbering sequence instead of independent numbers for Proposition
\newtheorem{proposition}[theorem]{Proposition}%
%%\newtheorem{proposition}{Proposition}% to get separate numbers for theorem and proposition etc.

%% \theoremstyle{thmstyletwo}%
\theoremstyle{remark}
\newtheorem{example}{Example}%
\newtheorem{remark}{Remark}%

%% \theoremstyle{thmstylethree}%
\theoremstyle{definition}
\newtheorem{definition}{Definition}%
\usepackage{amsmath}



\raggedbottom




% tightlist command for lists without linebreak
\providecommand{\tightlist}{%
  \setlength{\itemsep}{0pt}\setlength{\parskip}{0pt}}





\begin{document}


\title[Arriaga meets Kitagawa]{Arriaga meets Kitagawa. life expectancy
decompositions including population subgroups}

%%=============================================================%%
%% Prefix	-> \pfx{Dr}
%% GivenName	-> \fnm{Joergen W.}
%% Particle	-> \spfx{van der} -> surname prefix
%% FamilyName	-> \sur{Ploeg}
%% Suffix	-> \sfx{IV}
%% NatureName	-> \tanm{Poet Laureate} -> Title after name
%% Degrees	-> \dgr{MSc, PhD}
%% \author*[1,2]{\pfx{Dr} \fnm{Joergen W.} \spfx{van der} \sur{Ploeg} \sfx{IV} \tanm{Poet Laureate}
%%                 \dgr{MSc, PhD}}\email{iauthor@gmail.com}
%%=============================================================%%

\author*[1,2,3]{\pfx{Dr.} \fnm{Timothy} \sur{Riffe} }\email{\href{mailto:tim.riffe@ehu.eus}{\nolinkurl{tim.riffe@ehu.eus}}}



  \affil*[1]{\orgdiv{Department of Sociology and Social
Work}, \orgname{University of the Basque Country
(UPV/EHU)}, \orgaddress{\city{Leioa}, \country{Spain}, \postcode{48940}, \state{Bizkaia}, \street{Barrio
Sarriena s/n}}}
  \affil[2]{\orgname{Ikerbasque (Basque Foundation for Science)}}
  \affil[3]{\orgdiv{Laboratory of Population Health}, \orgname{Max
Planck Institute for Demographic Research}}

\abstract{\textbf{Background}: An Arriaga (1984) decomposition allows us
to decompose differences in life expectancy into contributions due to
mortality rate differences in each age. A Kitawaga (1955) decomposition
allows us to decompose differences in a weighted mean into an effect
from differences in structure and effects from differences in each
element of the weighted value.

\textbf{Objective:} Sometimes we would like to decompose a difference
between two populations that are each composed of like-defined
subpopulations. Said decomposition would produce effects for each age of
each subpopulation, as well as a marginal effect of composition
differences. I propose a straightforward analytic method to do this.

\textbf{Methods:} In short, within-subpopulation life expectancy
differences can be handled by the Arriaga method. The case of weighting
together the life expectancies of subpopulations by way of a composed
radix can be handled using the well-known Kitagawa method. The elements
of the value component of the Kitagawa method tell us how to rescale the
Arriaga results specific to each subpopulation.

\textbf{Results:} I show the proposed analytic method to be equivalent
to a Horiuchi et al (2008) reframing of the same problem. Some
mentionable properties: (i) There is no limit to the number of
subpopulations, (ii) it is straightforward to incorporate cause-of-death
information, (iii) composition is here only considered in the radix age.
I currently have results for simulated mortality rates, but I promise to
wrangle up an empirical application to demonstrate the method.

\textbf{Conclusions:} The analytic decomposition method I propose is
advantageous compared to a Horiuchi method for this problem purely for
reasons of computational efficiency. This method could help further
disentangle the effects of mortality and composition differences in
explaining or clarifying paradoxes or secular change. I promise to think
further about subpopulation weighting that might occur over all ages
(using prevalence information).\}}

\keywords{Decomposition, Mortality, Cause of death, Population
Structure, Mortality Inequalities}



\maketitle

\hypertarget{introduction}{%
\section{Introduction}\label{introduction}}

An \citet{arriaga1984measuring} decomposition allows us to decompose
changes in life expectancy into contributions due to mortality changes
in different ages. The method was designed to be practical, and framed
in terms of age lifetable columns expressed in discrete age groups. A
well-known property of the method is that mortality changes in different
ages need not be proportional. Derived contributions also sum exactly to
the observed life expectancy difference. A not-well-known property of
the method is that it is asymmetrical, in the sense that the absolute
values of age-specific contributions depend on whether we compare
population 1 with 2, or population 2 with 1. A further property of the
method is that it is designed to work with homogeneous populations,
meaning that populations 1 and 2 are each homogeneous, in the sense that
each is expressed with only one lifetable.

A \citet{kitagawa1955components} decomposition allows us to decompose
differences in a weighted mean due to differences in weights (structure)
and differences in the value being weighted (often rates, but in our
case life expectancies). This widely-used decomposition method is
well-known to be exact in that the resulting structure and value
components sum exactly to the observed difference in weighted means. The
individual elements (ages, or life expectancies for us) of the value
being weighted have identifiable effects. It is not well-known that the
individual elements of the structure component do not have identifiable
effects. Rather, the structure effects should be summed to a a marginal
effect due to differences in structure.

Sometimes we have a situation where we would like to decompose a
difference between two populations that are each composed of
like-defined subpopulations. For example, life expectancy in France
versus Spain, each with education-specific subpopulations. In such
situations, a decomposition should tell us the contribution to the
difference in overall life expectancy due to rate differences in each
age in each subpopulation, and also separate an effect due to overall
compositional change. In this paper, I would like to propose a
straightforward analytic method to decompose in this way. In short,
within-subpopulation life expectancy differences can be handled by the
Arriaga method. The case of weighting together the life expectancies of
subpopulations by way of a composed radix can be handled using the
well-known Kitagawa method. The elements of the value component of the
Kitagawa method tell us how to rescale the Arriaga results specific to
each subpopulation so as to isolate the age-subpopulation-specific
effects on the overall life expectancy difference. I will justify this
rescaling, and show that the results of this procedure are fully
consistent with the results of a \citet{horiuchi2008decomposition}
reframing of the problem.

\hypertarget{method}{%
\section{Method}\label{method}}

\hypertarget{notation}{%
\subsection{Notation}\label{notation}}

In formulas, we use the following variables and scripting, most of which
are lifetable columns:

\begin{itemize}

\item{$\ell(a)$} lifetable survivorship at exact age $a$.

\item{${}_nL_a$} lifetable person years lived in the interval $[a,a+n)$.

\item{$T_a$} total lifetable person years lived beyond age $a$.

\item{$e_a$} remaining life expectancy at exact age $a$.

\item{$\pi^s$} the initial population fraction for subgroup $s$ at the initial age. 
\end{itemize}

We also use the superscript \(s\) to index subpopulations comprising the
total population, and the superscript \(t\) to index time points, or
some other way of differentiating the total population. The index
\(e^{s,t}_0\) reads as ``life expectancy at birth for subpopulation
\(s\) at time \(t\)'', e.g.~\(e^{low, 1990}_0\) could be the life
expectancy in 1990 of a low education group.

\hypertarget{averaging-life-expectancy}{%
\subsection{Averaging life expectancy}\label{averaging-life-expectancy}}

There are two major approaches to producing average life expectancy for
a total population composed of subgroups, as \citet{vaupel2002life}
points out. One could derive the average mortality rate in each age by
summing the respective deaths and exposures from subpopulations and then
deriving the rate, as the Human Mortality Database
\citet{wilmoth2021methods} does, then recalculate the total population
lifetable from these. This is what Vaupel called \emph{current rates}
approach, which ignores even controllable heterogeneity. The other
approach derives subgroup-specific lifetables, then weights life
expectancies together according to an initial mixing composition in
order to arrive at the total life expectancy. This is Vaupel's
\emph{current conditions} approach, at least in part. This second
approach is what is often done in multistate models (cite), but also
sometimes with standard lifetables, for example, when doing
between-within decompositions of variance or other similar summary
measures. Our decomposition method deals with this second case,
specifically, we have:

\begin{equation}
\label{eq_avg}
e_0 = \sum_s^S \pi^s \cdot e_0^s \quad \textrm{,}
\end{equation}

\noindent where the \(S\) total subgroup fractions comprise a mixing
distribution, \(1 = \sum_s^S \pi^s\).

\hypertarget{kitagawa-decomposition}{%
\subsection{Kitagawa decomposition}\label{kitagawa-decomposition}}

Eq \eqref{eq_avg} treats the total life expectancy as a weighted
average, which means we can decompose a change over time in \(e_0\)
(\(\Delta = e_0^2 - e_o^1\)) exactly using the formulas from
\citet{kitagawa1955components}. Specifically, eq \eqref{eq_kit_comp}
gives an overall effect of changes in composition:

\begin{equation}
\label{eq_kit_comp}
\Delta^\text{composition} = \sum_s^S \left(\pi^{s,t2}- \pi^{s,t1}\right) \cdot \frac{e_0^{s,t1} + e_0^{s,t2}}{2} \quad \textrm{,}
\end{equation}

This result is widely known, but many are unaware that the composition
effect should be summed like this, since the group-specific composition
(structure) effects are not well-identified. Eq \eqref{eq_kit_e0} gives
the subgroup-specific effects from changes in life expectancy:

\begin{equation}
\label{eq_kit_e0}
\Delta^{e_0^s\text{~change}} = \left(e_0^{s,t2} - e_0^{s,t1}\right) \cdot \frac{\pi^{s,t1} + \pi^{s,t2}}{2}\quad \textrm{.}
\end{equation}

Eq \eqref{eq_kit_sum} says that the overall change in life expectancy is
the sum of (i) a single component that captures the effect of
composition change and (ii) a set of components describing the effect on
overall life expectancy from each group's specific life expectancy
change. This second effect could just as well be called the \emph{rate}
effect, since each life expectancy is fully determined by mortality
rates.

\begin{equation}
\label{eq_kit_sum}
\Delta = \Delta^\text{composition}  + \sum_s^S \Delta^{e_0^s\text{~change}}
\end{equation}

\hypertarget{symmetrical-arriaga-decomposition}{%
\subsection{Symmetrical Arriaga
decomposition}\label{symmetrical-arriaga-decomposition}}

The above-mentioned \emph{rate} effect can be conceived of as the
\emph{net} effect on total life expectancy change of the age-specific
effects within subgroups, as isolated by the many varieties of life
expectancy decompositions. In this case, it does not matter much which
method we choose to derive age-specific effects for subgroup-specific
changes in life expectancy. The \citet{arriaga1984measuring}
decomposition approach is popular in part because it is efficient,
analytic, and sums exactly to the observed life expectancy difference.
This method can be implemented in a number of relevant ways, per Riffe
et. al.~(2024), from which we here consider only \emph{symmetrical}
Arriaga decomposition. This implies following the original formulas
twice, once in each direction, and averaging the sign-adjusted results.
We repeat this exercise for each subgroup to decompose group-specific
changes in life expectancy (hence the \(^s\) on everything):

\begin{align}
\label{eq_kit_e0}
\Delta^s &= e_0^{s,2} - e_0^{s,1}  \\
 &= \sum _x^\omega {}_n\Delta_x^s \quad \textrm{,}\nonumber 
\end{align}

\noindent where \(\omega\) is the highest (open) age group, and
\(\Delta^s\) is the subgroup-specific (\(s\)) difference in life
expectancy being decomposed, and it is composed of age-specific
contributions, \(\overrightarrow{{}_n\Delta_x^s}\) or
\(\overleftarrow{{}_n\Delta_x^s}\) depending on whether we decompose
forward in time or backward, respectively (Riffe et. al., 2024). The
forward age-specific values \(\overrightarrow{{}_n\Delta_x^s}\) can be
calculated following Arriaga's decomposition method following eq
\eqref{eq_arriaga1}, consistent with \citet{arriaga1984measuring} or the
presentation in \citet{preston2000demography}. I use a lifetable radix
of 1, meaning \(\ell_0 = 1\), to reduce the formula slightly.

\begin{equation}
\label{eq_arriaga1}
 \overrightarrow{{}_n\Delta_x^s} = \begin{cases}\ell_x^{s,1} \cdot \left( \frac{{}_nL_x^{s,2}}{\ell_x^{s,2}} - \frac{{}_nL_x^{s,1}}{\ell_x^{s,1}}\right)+T_{x+n}^{s,2}\cdot\left(\frac{\ell_x^{s,1}}{\ell_x^{s,2}} - \frac{\ell_{x+n}^{s,1}}{\ell_{x+n}^{s,2}}\right)  & \text{$ \forall x < \omega$}  \\
  \ell_\omega ^{s,1} \cdot (e_\omega^{s,2} -e_\omega^{s,1} ) & \text{$\forall x = \omega $}
  \end{cases}
\end{equation}

Equation \eqref{eq_arriaga1} is the first pass of our symmetrical
decomposition, and \eqref{eq_arriaga2} is the second pass, identical
except all time superscripts are swapped.

\begin{equation}
\label{eq_arriaga2}
 \overleftarrow{{}_n\Delta_x^s} = \begin{cases}\ell_x^{s,2} \cdot \left( \frac{{}_nL_x^{s,1}}{\ell_x^{s,1}} - \frac{{}_nL_x^{s,2}}{\ell_x^{s,2}}\right)+T_{x+n}^{s,1}\cdot\left(\frac{\ell_x^{s,2}}{\ell_x^{s,1}} - \frac{\ell_{x+n}^{s,2}}{\ell_{x+n}^{s,1}}\right)  & \text{$ \forall x < \omega$}  \\
  \ell_\omega ^{s,2} \cdot (e_\omega^{s,1} -e_\omega^{s,2} ) & \text{$\forall x = \omega $}
  \end{cases}
\end{equation}

Importantly,

\begin{align}
\label{eq_sum2}
 -\Delta^s &= \sum_x^\omega \overleftarrow{{}_n\Delta_x^s} \\
 &= e_0^1 - e_0^2 \nonumber
\end{align}

Then our symmetrical estimate of \({}_n\Delta_x^s\) is just the average
of these:

\begin{equation}
\label{eq_arriaga3}
{}_n\Delta_x^s = \frac{\left(\overrightarrow{{}_n\Delta_x^s} -  \overleftarrow{{}_n\Delta_x^s}\right)}{2}
\end{equation}

Repeat the above to derive \({}_n\Delta_x^s\) for each
within-subpopulation comparison.

\hypertarget{rescale-arriaga-results}{%
\subsection{Rescale Arriaga results}\label{rescale-arriaga-results}}

The quantity \({}_n\Delta_x^s\) given in eq \eqref{eq_arriaga3} is on
the scale of group \(s\) specific life expectancy changes, but to know
its net impact on overall life expectancy change we rescale per eq
\eqref{eq_rescale} to match the effect sizes from eq \eqref{eq_kit_e0}

\begin{equation}
\label{eq_rescale}
{}_n\Delta_x^{s,\text{net effect}} = \Delta^{e_0^s\text{~change}} \cdot \frac{{}_n\Delta_x^s}{\Delta^s}
\end{equation}

\backmatter

\bmhead{Supplementary information}

If your article has accompanying supplementary file/s please state so
here.

\bmhead{Acknowledgments}

Acknowledgments are not compulsory. Where included they should be brief.
Grant or contribution numbers may be acknowledged.

Please refer to Journal-level guidance for any specific requirements.

\hypertarget{declarations}{%
\section*{Declarations}\label{declarations}}
\addcontentsline{toc}{section}{Declarations}

Some journals require declarations to be submitted in a standardised
format. Please check the Instructions for Authors of the journal to
which you are submitting to see if you need to complete this section. If
yes, your manuscript must contain the following sections under the
heading `Declarations':

\begin{itemize}
\tightlist
\item
  Funding
\item
  Conflict of interest/Competing interests (check journal-specific
  guidelines for which heading to use)
\item
  Ethics approval
\item
  Consent to participate
\item
  Consent for publication
\item
  Availability of data and materials
\item
  Code availability
\item
  Authors' contributions
\end{itemize}

\noindent If any of the sections are not relevant to your manuscript,
please include the heading and write `Not applicable' for that section.

\begin{appendices}

\hypertarget{secA1}{%
\section{Section title of first appendix}\label{secA1}}

An appendix contains supplementary information that is not an essential
part of the text itself but which may be helpful in providing a more
comprehensive understanding of the research problem or it is information
that is too cumbersome to be included in the body of the paper.

For submissions to Nature Portfolio Journals please use the heading
``Extended Data''.

\end{appendices}

\bibliography{bibliography.bib}


\end{document}
